\documentclass[12pt,a4paper]{article}
\usepackage{helvet}
\usepackage{graphicx}
\usepackage{fancyhdr}
\usepackage[hidelinks]{hyperref}
\usepackage{url}

% Break URL on hyphens
\makeatletter
\g@addto@macro{\UrlBreaks}{\UrlOrds}
\makeatother

\pagestyle{fancy}
\fancyhf{}
\renewcommand{\headrulewidth}{0pt} % Remove line
\fancyfoot[R]{Version \input{|"git rev-parse --short HEAD"}}
\fancyhead[C]{\leftmark}
\fancyfoot[L]{\thepage}

\setlength{\parindent}{0cm}  % No indent on pararaph

\title{VHDL 101}
\author{Patrick Mintram}

\begin{document}

\clearpage\maketitle
\thispagestyle{empty} % No page numbering
\pagebreak

\section{Intoduction}
Things this session is intended to wet your appetite to the world of FPGAs, so that if you choose to you can start having a play in your own time. This session is:
\begin{enumerate}
    \item A brief overview of VHDL
    \item A chance to get hands on with some Hardware
    \item A chance to make a 'hello world' in Hardware
\end{enumerate}

Things this session is not:
\begin{enumerate}
    \item An introduction to Digital Design
    \item A comprehensive deep dive into VHDL
    \item Likely to finish on time
\end{enumerate}

\pagebreak
\tableofcontents
\listoffigures
\pagebreak

\setlength{\parskip}{1em}

\section{Why Should I Care?}
FPGAs enable Low Latency processing \footnote{\url{https://blog.esciencecenter.nl/why-use-an-fpga-instead-of-a-cpu-or-gpu-b234cd4f309c}}, so performing a transform on data coming in and getting the result output an be much faster than in a traditional CPU based approach. They also provide far more IO configurability than the traditional approach; the IO logic, and the pin it's connected to are totally configurable in code and contstraints files. Say a rquirement changes from an 8 bit UART bus to a proprietry 11 but UART bus - this would require a whole new microcontroller in a traditional approach however with an FPGA this might only require a change to a \texttt{generic} and a recompile. 

For the reasons stated above, typical uses include signal processing such as filtering \footnote{\url{https://digital-library.theiet.org/content/journals/10.1049/iet-cdt.2016.0067}}, and High Speed IO such as devices produced by SpeedGoat \footnote{\url{https://www.speedgoat.com/products/simulink-programmable-fpgas-fpga-i-o-modules-io334}}.

\section{How to use this guide}
Coloured boxes for step by step instructions. Not expected to follow it step by step but just reference it to work at own pace.

\subsection{Terminology}
HDL
Entity
Module
Architecture
Dataflow
RTL
Sythesise

\subsection{Toolchain}
Vivade vs Quartus. Others - yosys, ghdl etc
\section{Important things to remember}
It's not software it's hardware
\section{Finally lets get to doing some VHDL}

\subsection{Reference Project}
This will be an overview of the refernce project
\subsection{How to see output from our VHDL}
\subsubsection{Simulation}
How to interpret the waveforms
\subsubsection{On the hardware itself}
How to load onto the board



\end{document}
